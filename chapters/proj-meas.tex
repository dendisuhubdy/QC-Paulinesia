\chapter{One and Two Qubit Projective Measurements}

\claim (Conversion: 1 qubit internal measurement $\rarrow$
1 qubit final measurement)

For any $j\in Bool$,

\grayeq{
\beq
\begin{array}{c}
\Qcircuit @C=1em @R=1em @!R{
&
&
\\
&\gate{\ket{j}\bra{j}}
&\qw
}
\end{array}
=
\begin{array}{c}
\Qcircuit @C=1em @R=.5em @!R{
\freegate{\bra{j}}
&\timesgate\qwx[1]
&\gate{\ket{0}}
\\
&\dotgate
&\qw
}
\end{array}
\;.
\label{eq-1qbit-proj}
\eeq}
\proof
Let LHS and RHS stand for the left and
right hand sides of Eq.(\ref{eq-1qbit-proj}).
For any $b\in Bool$,

\beq
LHS\ket{b}_\bitb =
\ket{b}_\bitb \delta_j^b
\;.
\eeq

\beqa
RHS\ket{b}_\bitb &=&
\bra{j}_\bita
\cnot{\bitb}{\bita}
\ket{0,b}_{\bita\bitb}\\
&=&
\bra{j}_\bita \ket{b,b}_{\bita\bitb}\\
&=&
\ket{b}_\bitb \delta_j^b
\;.
\eeqa
\qed

One qubit operations (such as
internal or final one qubit measurements
or one qubit rotations)
are ``cheap" compared with two qubit operations
such as CNOTs and two qubit measurements (either
internal or final). This is because two
qubit operations are slower and
they require two qubits to interact,
which  opens the door for noise from
the environment to creep in.
So in this section we
will pay attention only to
the number of two qubit operations.
Let {\bf bibit} stand for two bits.
Next we will show what I like to
call the ``one to
two" conversion rules. Namely, given a single
CNOT, one can always convert it to two
bibit operations. Likewise,
given a single bibit operation, one can always
convert it to two CNOTs.

As usual in this document,
for $j\in Bool$, we define
$\pizz{j}(\bita, \bitb)=
\pi[\sigzz(\bita, \bitb)=(-1)^j]$; i.e,
$\pizz{j}$ is
the projection operator onto the
2 qubit subspace with $(-1)^j$
as eigenvalue for $\sigz\otimes\sigz$.


\claim (Conversion: 1 bibit
measurement $\rarrow$
2 CNOTs)

For any $j\in Bool$,

\grayeq{
\beq
\begin{array}{c}
\Qcircuit @C=1em @R=1.5em @!R{
&\multigate{1}{\pizz{j}}
&\qw
\\
&\ghost{\pizz{j}}
&\qw
}
\end{array}
=
\begin{array}{c}
\Qcircuit @C=1em @R=.5em @!R{
&\dotgate\qwx[1]
&\qw
&\dotgate\qwx[1]
&\qw
\\
&\timesgate
&\gate{\ket{j}\bra{j}}
&\timesgate
&\qw
}
\end{array}
\;.
\label{eq-1meas-to-2cnots-2bit}
\eeq}
\proof
Let RHS stand for the right
hand side of Eq.(\ref{eq-1meas-to-2cnots-2bit}).
For any $a,b\in Bool$,


\beqa
RHS\ket{a,b}_{\bita\bitb}&=&
\cnot{\bita}{\bitb}
\ket{j}_\bitb
\bra{j}_\bitb
\cnot{\bita}{\bitb}
\ket{a,b}_{\bita\bitb}\\
&=&
\cnot{\bita}{\bitb}
\ket{j}_\bitb
\delta^j_{a\oplus b}
\ket{a}_\bita\\
&=&
\delta^j_{a\oplus b}
\ket{a,b}_{\bita\bitb}\\
&=&
\pizz{j}\ket{a,b}_{\bita\bitb}
\;.
\eeqa
\qed

\claim (Conversion: 1 bibit measurement $\rarrow$
1 CNOT. Special case of
Eq.(\ref{eq-1meas-to-2cnots-2bit}).)

For any $j,k\in Bool$,

\grayeq{
\beq
\begin{array}{c}
\Qcircuit @C=1em @R=.5em @!R{
\freegate{\bra{k}}
&\multigate{1}{\pizz{j}}
&\qw
\\
&\ghost{\pizz{j}}
&\qw
}
\end{array}
=
\begin{array}{c}
\Qcircuit @C=1em @R=.5em @!R{
\freegate{\bra{k}}
&\qw
&\qw
&\dotgate\qwx[1]
&\qw
\\
&\gate{\sigx^k}
&\gate{\ket{j}\bra{j}}
&\timesgate
&\qw
}
\end{array}
\;.
\label{eq-1meas-to-1cnot}
\eeq}
\proof
Follows immediately from
Eq.(\ref{eq-1meas-to-2cnots-2bit}).
\qed

\claim (Another Conversion of: 1 bibit
measurement $\rarrow$
2 CNOTs)

For any $j\in Bool$,

\grayeq{
\beq
\begin{array}{c}
\Qcircuit @C=1em @R=1.5em @!R{
&
&
\\
&\multigate{1}{\pizz{j}}
&\qw
\\
&\ghost{\pizz{j}}
&\qw
}
\end{array}
=
\begin{array}{c}
\Qcircuit @C=1em @R=.5em @!R{
\freegate{\bra{j}}
&\timesgate\qwx[1]
&\timesgate\qwx[2]
&\gate{\ket{0}}
\\
&\dotgate
&\qw
&\qw
\\
&\qw
&\dotgate
&\qw
}
\end{array}
\;.
\label{eq-1meas-to-2cnots-3bit}
\eeq}
\proof
Let LHS and RHS denote the left and
right hand sides of
Eq.(\ref{eq-1meas-to-2cnots-3bit}).

\beqa
LHS &=&
\begin{array}{c}
\Qcircuit @C=1em @R=.5em @!R{
&
&
&
&
\\
&\dotgate\qwx[1]
&\qw
&\dotgate\qwx[1]
&\qw
\\
&\timesgate
&\gate{\ket{j}\bra{j}}
&\timesgate
&\qw
}
\end{array}\\
&=&
\begin{array}{c}
\Qcircuit @C=1em @R=.5em @!R{
\freegate{\bra{j}}
&\qw
&\timesgate\qwx[2]
&\qw
&\gate{\ket{0}}
\\
&\dotgate\qwx[1]
&\qw
&\dotgate\qwx[1]
&\qw
\\
&\timesgate
&\dotgate
&\timesgate
&\qw
}
\end{array}\\
&=&
\begin{array}{c}
\Qcircuit @C=1em @R=.5em @!R{
\freegate{\bra{j}}
&\timesgate\qwx[1]
&\timesgate\qwx[2]
&\qw
&\qw
&\gate{\ket{0}}
\\
&\dotgate
&\qw
&\dotgate\qwx[1]
&\dotgate\qwx[1]
&\qw
\\
&\qw
&\dotgate
&\timesgate
&\timesgate
&\qw
}
\end{array}\\
&=&RHS
\;.
\eeqa
\qed

\claim (Conversion: 1 CNOT $\rarrow$
2 bibit measurements)

For any $k, j_1, j_2\in Bool$,

\grayeq{
\beq
\begin{array}{c}
\Qcircuit @C=1em @R=2em @!R{
&\dotgate\qwx[2]
&\qw
\\
&
&
\\
&\timesgate
&\qw
}
\end{array}
=
(-1)^{(k+j_1)j_2}2\sqrt{2}
\begin{array}{c}
\Qcircuit @C=.5em @R=.5em @!R{
&\gate{\sigz^{j_2}}
&\qw
&\qw
&\qw
&\multigate{1}{\pizz{j_1}}
&\qw
&\qw
\\
\freegate{\bra{k}}
&\qw
&\gate{H}
&\multigate{1}{\pizz{j_2}}
&\gate{H}
&\ghost{\pizz{j_1}}
&\gate{H}
&\gate{\ket{0}}
\\
&\gate{\sigx^{k+j_1}}
&\gate{H}
&\ghost{\pizz{j_2}}
&\gate{H}
&\qw
&\qw
&\qw
}
\end{array}
\;.
\label{eq-1cnot-to-2meas}
\eeq}
\proof
Define $T$ by

\beq
T=
\begin{array}{c}
\Qcircuit @C=.5em @R=.5em @!R{
&\qw
&\qw
&\qw
&\multigate{1}{\pizz{j_1}}
&\qw
&\qw
\\
\freegate{\bra{k}}
&\gate{H}
&\multigate{1}{\pizz{j_2}}
&\gate{H}
&\ghost{\pizz{j_1}}
&\gate{H}
&\gate{\ket{0}}
\\
&\gate{H}
&\ghost{\pizz{j_2}}
&\gate{H}
&\qw
&\qw
&\qw
}
\end{array}
\;.
\eeq
Then

\beq
T=
\underbrace{
\begin{array}{c}
\Qcircuit @C=.5em @R=.5em @!R{
&\qw
&\qw
&\qw
\\
\freegate{\bra{k}}
&\gate{H}
&\timesgate\qwx[1]
&\gate{\ket{j_2}}
\\
&\gate{H}
&\dotgate
&\qw
}
\end{array}
}_{T_1}
\;
\underbrace{
\begin{array}{c}
\Qcircuit @C=.5em @R=.5em @!R{
&\qw
&\qw
&\dotgate\qwx[1]
&\qw
\\
\freegate{\bra{j_2}}
&\timesgate\qwx[1]
&\gate{H}
&\timesgate
&\gate{\ket{j_1}}
\\
&\dotgate
&\gate{H}
&\qw
&\qw
}
\end{array}
}_{T_2}
\;
\underbrace{
\begin{array}{c}
\Qcircuit @C=.5em @R=.5em @!R{
&\dotgate\qwx[1]
&\qw
&\qw
&\qw
\\
\freegate{\bra{j_1}}
&\timesgate
&\gate{H}
&\gate{\ket{0}}
\\
&\qw
&\qw
&\qw
}
\end{array}
}_{T_3}
\;\;,
\eeq

\beq
T_1 =
\frac{(-1)^{k j_2}}{\sqrt{2}}
H(\bitc)
\sigz^k(\bitc)
\;,
\eeq

\beq
T_3 = \frac{1}{\sqrt{2}}
\;,
\eeq

\beqa
T_2 &=&
\begin{array}{c}
\Qcircuit @C=1em @R=.5em @!R{
&\qw
&\qw
&\dotgate\qwx[1]
&\qw
\\
\freegate{\bra{j_2}}
&\gate{H}
&\dotgate\qwx[1]
&\timesgate
&\gate{\ket{j_1}}
\\
&\gate{H}
&\timesgate
&\qw
&\qw
}
\end{array}\\
&=&
\begin{array}{c}
\Qcircuit @C=1em @R=.5em @!R{
&\qw
&\dotgate\qwx[2]
&\dotgate\qwx[1]
&\qw
&\qw
\\
\freegate{\bra{j_2}}
&\gate{H}
&\qw
&\timesgate
&\dotgate\qwx[1]
&\gate{\ket{j_1}}
\\
&\gate{H}
&\timesgate
&\qw
&\timesgate
&\qw
}
\end{array}\\
&=&
\frac{(-1)^{j_1 j_2}}{\sqrt{2}}
\begin{array}{c}
\Qcircuit @C=1em @R=.5em @!R{
&\qw
&\dotgate\qwx[2]
&\gate{\sigz^{j_2}}
\\
&
&
&
\\
&\gate{H}
&\timesgate
&\gate{\sigx^{j_1}}
}
\end{array}
\;.
\eeqa
Putting all this together,

\beqa
T &=& T_1 T_2 T_3\\
&=&
\frac{(-1)^{(k+j_1) j_2}}{2\sqrt{2}}
\begin{array}{c}
\Qcircuit @C=1em @R=.5em @!R{
&\qw
&\dotgate\qwx[1]
&\gate{\sigz^{j_2}}
\\
&\gate{H\sigz^k H}
&\timesgate
&\gate{\sigx^{j_1}}
}
\end{array}\\
&=&
\frac{(-1)^{(k+j_1) j_2}}{2\sqrt{2}}
\begin{array}{c}
\Qcircuit @C=1em @R=.5em @!R{
&\gate{\sigz^{j_2}}
&\dotgate\qwx[1]
&\qw
\\
&\gate{\sigx^{k+j_1}}
&\timesgate
&\qw
}
\end{array}
\;.
\eeqa
\altproof
Define operator $S$ such that for all
$a,c\in Bool$,
\beq
S\ket{a,c}_{\bita\bitc}=
\bra{k}_\bitb H(\bitb)
\pizz{j_2}(\bitb,\bitc)
H(\bitb)
\pizz{j_1}(\bita, \bitb)
H(\bitb)
\ket{a,0,c}_{\bita\bitb\bitc}
\;.
\label{eq-alg-t}
\eeq
In Eq.(\ref{eq-alg-t}),
insert a partition of unity
$\sum_{(a_1, b_1, c_1)\in Bool^3}
\ket{a_1,b_1, c_1}
\bra{a_1,b_1, c_1}$
before the first bibit
measurement and
another
$\sum_{(a_2, b_2, c_2)\in Bool^3}
\ket{a_2,b_2, c_2}
\bra{a_2,b_2, c_2}$
before the second.
Then use the fact that
for $a,b,j\in Bool$,
$\pizz{j}\ket{a,b} = \delta^j_{a\oplus b}\ket{a,b}$.
Details left to the reader.
\qed

\claim (Conversion: 1 CNOT $\rarrow$ 1
bibit measurement.
Special case of
Eq.(\ref{eq-1cnot-to-2meas}).)

For any $j,k\in Bool$,

\grayeq{
\beq
\begin{array}{c}
\Qcircuit @C=1em @R=.5em @!R{
&\dotgate\qwx[1]
&\qw
\\
&\timesgate
&\gate{\ket{j}}
}
\end{array}
=
(-1)^{jk}\sqrt{2}
\begin{array}{c}
\Qcircuit @C=1em @R=.5em @!R{
&\gate{\sigz^j}
&\multigate{1}{\pizz{j}}
&\qw
&\qw
\\
&\qw
&\ghost{\pizz{j}}
&\gate{H}
&\gate{\ket{k}}
}
\end{array}
\;.
\label{eq-1cnot-1meas}
\eeq}
\proof
Let LHS and RHS stand for the left and
right hand sides of Eq.(\ref{eq-1cnot-1meas}).
Then

\beq
RHS=
(-1)^{jk}\sqrt{2}
\begin{array}{c}
\Qcircuit @C=1em @R=.5em @!R{
&\gate{\sigz^j}
&\dotgate\qwx[1]
&\qw
&\dotgate\qwx[1]
&\qw
&\qw
\\
&\qw
&\timesgate
&\gate{\ket{j}\bra{j}}
&\timesgate
&\gate{H}
&\gate{\ket{k}}
}
\end{array}
=
LHS
\;.
\eeq
\qed
