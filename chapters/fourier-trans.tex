\chapter{Quantum Fourier Transform}

For this section, it is especially
important that
the reader read the Notation section
of QC Paulinesia.
The Notation section explains
what we mean by natural labelling.
Natural labelling will
be used in this section.


Given a vector
$\vec{x}=(x_{\nb-1},\ldots, x_1, x_0 )\in Bool^\nb$,
let
$R\vec{x}=(x_0, x_1, \ldots,x_{\nb-1})$.
Thus $R$ is the matrix that reverses the components
of an $\nb$ dimensional vector.
For example, for $\nb=4$,

\beq
R =
\left(
\begin{array}{cccc}
&&&1\\
\text{\huge 0}&&1& \\
&1&& \\
1&&&\text{\huge 0} \\
\end{array}
\right)
\;.
\eeq
We will also use $R$ to denote
a map from the Hilbert space of $\nb$
qubits to itself such that $R\ket{\vec{x}}=
\ket{R\vec{x}}$ for $\vec{x}\in Bool^\nb$.
We will also use $R$ to denote
the map $R:Z_{0,\nb-1}\rarrow Z_{0,\nb-1}$
such that $R(i)=\nb-1-i$. For example, for
$\nb=4$,
$R$  maps $0\rarrow 3$, $1\rarrow 2$,
$2\rarrow 1$, $3\rarrow 0$.

For any $\bita, \bitb \in Z_{0,\nb-1}$, define

\grayeq{
\beq
\begin{array}{c}
\Qcircuit @C=1em @R=1.5em @!R{
&\dotgate\qwx[1]
&\qw
\\
&\dotgate
&\qw
}
\end{array}
=
V(\bita, \bitb)=
\exp[ i\pi
\frac{n(\bita)n(\bitb)}{2^{|\bita-\bitb|}}
]
=
(-1)^\frac{n(\bita)n(\bitb)}{2^{|\bita-\bitb|}}
\;.
\label{eq-fou-v-def}
\eeq}
Note that normally in QC Paulinesia, we use
$
\begin{array}{c}
\Qcircuit @C=1em @R=1.5em @!R{
&\dotgate\qwx[1]
&\qw
\\
&\dotgate
&\qw
}
\end{array}
=\sigz^{n(\bita)}(\bitb)=
(-1)^{n(\bita)n(\bitb)}
\;,
$
so the definition given by Eq.(\ref{eq-fou-v-def})
applies only to this section.


For any $x\in Z_{0, \ns-1}$, the
{\bf Quantum Fourier Transform}
of $\ket{x}$ is defined by

\grayeq{
\beq
U_{FT}\ket{x}=
\frac{1}{\sqrt{\ns}}
\sum_{y=0}^{\ns-1}
e^{ i\frac{2\pi xy}{\ns} }
\ket{y}
\;.
\eeq}


Henceforth, for simplicity,
we will often assume $\nb=4$.
It will be obvious how to extend our
arguments to other values of $\nb$.



\claim

For any
$\vec{x}=(x_3,x_2,x_1, x_0 )\in Bool^4$,

\grayeq{
\beq
\begin{array}{c}
\Qcircuit @C=1em @R=.5em @!R{
&\freemultigate{3}{U_{FT}}
&\gate{\ket{x_0}}
\\
&\freeghost{U_{FT}}
&\gate{\ket{x_1}}
\\
&\freeghost{U_{FT}}
&\gate{\ket{x_2}}
\\
&\freeghost{U_{FT}}
&\gate{\ket{x_3}}
}
\end{array}
 =
\begin{array}{c}
\Qcircuit @C=1em @R=.5em @!R{
&\qw
&\qw
&\qw
&\qw
&\qw
&\qw
&\dotgate\qwx[3]
&\dotgate\qwx[2]
&\dotgate\qwx[1]
&\gate{H}
&\gate{\ket{x_3}}
\\
&\qw
&\qw
&\qw
&\dotgate\qwx[2]
&\dotgate\qwx[1]
&\gate{H}
&\qw
&\qw
&\dotgate
&\qw
&\gate{\ket{x_2}}
\\
&\qw
&\dotgate\qwx[1]
&\gate{H}
&\qw
&\dotgate
&\qw
&\qw
&\dotgate
&\qw
&\qw
&\gate{\ket{x_1}}
\\
&\gate{H}
&\dotgate
&\qw
&\dotgate
&\qw
&\qw
&\dotgate
&\qw
&\qw
&\qw
&\gate{\ket{x_0}}
}
\end{array}
\;,
\eeq}
\proof

Recall from the Notation section
that $\vec{\nu}=(\nb-1,\ldots,2 , 1, 0)$.
Let
$n=2^{\vec{\nu}}\cdot \vec{n}$ and
$x=2^{\vec{\nu}}\cdot \vec{x}$. Then


\beqa
U_{FT}\ket{\vec{x}}_{\vec{\nu}}&=&
\frac{1}{\sqrt{\ns}}
e^{ \frac{i 2\pi xn}{\ns} }
\sum_{\vec{y}\in Bool^\nb}
\ket{\vec{y}}_{\vec{\nu}}\\
&=&
e^{i \frac{2\pi xn}{\ns} }
H(\vec{\nu})
\ket{0}_{\vec{\nu}}
\;.
\eeqa
Furthermore,



\beqa
\exp[\frac{i 2\pi x n}{16} ]&=&
e^{\left[
\frac{i 2\pi }{16}
(8x_3 + 4x_2 + 2x_1 + x_0  )
(8n(3) + 4n(2) + 2n(1) + n(0))
\right]}\\
&=&
\exp[i 2\pi
\left\{
\begin{array}{l}
\;\;\;n(3)(\frac{x_0}{2})\\
+n(2)(\frac{x_1}{2}+\frac{x_0}{4})\\
+n(1)(\frac{x_2}{2}+\frac{x_1}{4}+\frac{x_0}{8})\\
+n(0)(\frac{x_3}{2}+\frac{x_2}{4}+\frac{x_1}{8}+\frac{x_0}{16})
\end{array}
\right\} ]
\label{eq-fou-omit-two-pi}
\;,
\eeqa
where, in Eq.(\ref{eq-fou-omit-two-pi}),
we omitted all terms
in the argument of the exponential
that
yielded contributions of the form
$e^{i 2\pi (integer)}$.

Note that for any $x\in Bool$
and bit $\bita$,

\beq
(-1)^{xn(\bita)}H(\bita)\ket{0}_\bita=
\sigz^{x}(\bita)H(\bita)\ket{0}_\bita=
H(\bita)\ket{x}_\bita
\;.
\eeq
Thus,

\beqa
\lefteqn{\exp[\frac{i 2\pi x n}{16} ]
H(\vec{\nu})\ket{0}_{\vec{\nu}} =
\left\{
\begin{array}{l}
\exp[i \pi n(3) x_0]H(3)\ket{0}_3\\
\exp[i \pi n(2)(x_1+\frac{x_0}{2})]H(2)\ket{0}_2\\
\exp[i \pi n(1)(x_2+\frac{x_1}{2}+\frac{x_0}{4})]H(1)\ket{0}_1\\
\exp[i \pi n(0)(x_3+\frac{x_2}{2}+\frac{x_1}{4}+\frac{x_0}{8})]H(0)\ket{0}_0
\end{array}
\right.}
\\
&=&
\left\{
\begin{array}{l}
H(3)\ket{x_0}_3\\
\exp[i \pi n(2)(\frac{x_0}{2})]H(2)\ket{x_1}_2\\
\exp[i \pi n(1)(\frac{x_1}{2}+\frac{x_0}{4})]H(1)\ket{x_2}_1\\
\exp[i \pi n(0)(\frac{x_2}{2}+\frac{x_1}{4}+\frac{x_0}{8})]H(0)\ket{x_3}_0
\end{array}
\right.
\\
&=&
\left\{
\begin{array}{l}
H(3)\\
\exp[i \pi n(2)(\frac{n(3)}{2})]H(2)\\
\exp[i \pi n(1)(\frac{n(2)}{2}+\frac{n(3)}{4})]H(1)\\
\exp[i \pi n(0)(\frac{n(1)}{2}+\frac{n(2)}{4}+\frac{n(3)}{8})]H(0)
\end{array}
\right\}R\ket{\vec{x}}
\\
&=&
H(3)
V(3,2)
H(2)
V(3,1)V(2,1)
H(1)
V(3,0)V(2,0)V(1,0)
H(0)
R\ket{\vec{x}}
\;.
\eeqa
\qed

\claim(3-2-1 form equals 1-2-3 form)

\grayeq{
\beqa
U_{FT} &=&
\begin{array}{c}
\Qcircuit @C=1em @R=.5em @!R{
&\qw
&\qw
&\qw
&\qw
&\qw
&\qw
&\dotgate\qwx[3]
&\dotgate\qwx[2]
&\dotgate\qwx[1]
&\gate{H}
&\multigate{3}{R}
\\
&\qw
&\qw
&\qw
&\dotgate\qwx[2]
&\dotgate\qwx[1]
&\gate{H}
&\qw
&\qw
&\dotgate
&\qw
&\ghost{R}
\\
&\qw
&\dotgate\qwx[1]
&\gate{H}
&\qw
&\dotgate
&\qw
&\qw
&\dotgate
&\qw
&\qw
&\ghost{R}
\\
&\gate{H}
&\dotgate
&\qw
&\dotgate
&\qw
&\qw
&\dotgate
&\qw
&\qw
&\qw
&\ghost{R}
}
\end{array}
\label{eq-1-2-3}
\\
&=&
\begin{array}{c}
\Qcircuit @C=1em @R=.5em @!R{
&\qw
&\qw
&\qw
&\dotgate\qwx[3]
&\qw
&\qw
&\dotgate\qwx[2]
&\qw
&\dotgate\qwx[1]
&\gate{H}
&\multigate{3}{R}
\\
&\qw
&\qw
&\dotgate\qwx[2]
&\qw
&\qw
&\dotgate\qwx[1]
&\qw
&\gate{H}
&\dotgate
&\qw
&\ghost{R}
\\
&\qw
&\dotgate\qwx[1]
&\qw
&\qw
&\gate{H}
&\dotgate
&\dotgate
&\qw
&\qw
&\qw
&\ghost{R}
\\
&\gate{H}
&\dotgate
&\dotgate
&\dotgate
&\qw
&\qw
&\qw
&\qw
&\qw
&\qw
&\ghost{R}
}
\end{array}
\;.
\label{eq-3-2-1}
\eeqa}
\proof
Obvious.
\qed

We call ``the 1-2-3 form"
the form of $U_{FT}$ given
by Eq.(\ref{eq-1-2-3}).
We call ``the 3-2-1 form" the
form given by Eq.(\ref{eq-3-2-1}).
The
 numbers 1,2,3
refer to the
number of $V$ operators
between the $H$ operators.


\claim

$U_{FT}$ is a symmetric matrix.
\proof

Let $\dagger=$ Hermitian conjugate,
$*=$ complex conjugate, so $\dagger *=$ transpose.
For any $x,y\in Z_{0,\ns-1}$,

\beq
\bra{y} U_{FT} \ket{x} =
\exp(\frac{i2\pi x y}{\ns}) =
\bra{y} U_{FT}^{\dagger *} \ket{x}
\;.
\eeq
\altproof
\beqa
U_{FT}^{\dagger *} &=&
\begin{array}{c}
\Qcircuit @C=1em @R=.5em @!R{
\freemultigate{3}{R}
&\gate{H}
&\dotgate\qwx[1]
&\dotgate\qwx[2]
&\dotgate\qwx[3]
&\qw
&\qw
&\qw
&\qw
&\qw
&\qw
\\
\freeghost{R}
&\qw
&\dotgate
&\qw
&\qw
&\gate{H}
&\dotgate\qwx[1]
&\dotgate\qwx[2]
&\qw
&\qw
&\qw
\\
\freeghost{R}
&\qw
&\qw
&\dotgate
&\qw
&\qw
&\dotgate
&\qw
&\gate{H}
&\dotgate\qwx[1]
&\qw
\\
\freeghost{R}
&\qw
&\qw
&\qw
&\dotgate
&\qw
&\qw
&\dotgate
&\qw
&\dotgate
&\gate{H}
}
\end{array}\\
&=&
\begin{array}{c}
\Qcircuit @C=1em @R=.5em @!R{
&\qw
&\qw
&\qw
&\dotgate\qwx[3]
&\qw
&\qw
&\dotgate\qwx[2]
&\qw
&\dotgate\qwx[1]
&\gate{H}
&\multigate{3}{R}
\\
&\qw
&\qw
&\dotgate\qwx[2]
&\qw
&\qw
&\dotgate\qwx[1]
&\qw
&\gate{H}
&\dotgate
&\qw
&\ghost{R}
\\
&\qw
&\dotgate\qwx[1]
&\qw
&\qw
&\gate{H}
&\dotgate
&\dotgate
&\qw
&\qw
&\qw
&\ghost{R}
\\
&\gate{H}
&\dotgate
&\dotgate
&\dotgate
&\qw
&\qw
&\qw
&\qw
&\qw
&\qw
&\ghost{R}
}
\end{array}\\
&=& U_{FT}
\;.
\eeqa
\qed

For distinct bits $\bita, \bitb$,
let $V(\bita, \bitb)_{n(\bitb)\rarrow b}$
denote the result of substituting $n(\bitb)$
in
$V(\bita, \bitb)$
by $b\in Bool$.

\claim

For any
$\vec{x}=(x_3,x_2,x_1, x_0 )\in Bool^4$
and $\vec{y}=(y_3,y_2,y_1, y_0 )\in Bool^4$,

\grayeq{\small
\beq
\bra{\vec{y}}U_{FT}\ket{\vec{x}}=
\begin{array}{c}
\Qcircuit @C=1em @R=.5em @!R{
&&&
&\freegate{\bra{y_0}H(0)}
&\gate{\ket{x_{R(0)}}}
\\
&&
&\freegate{\bra{y_1}H(1)}
&\gate{V(1,0)_{n(0)\rarrow y_0}}
&\gate{\ket{x_{R(1)}}}
\\
&
&\freegate{\bra{y_2}H(2)}
&\gate{V(2,1)_{n(1)\rarrow y_1}}
&\gate{V(2,0)_{n(0)\rarrow y_0}}
&\gate{\ket{x_{R(2)}}}
\\
&\freegate{\bra{y_3}H(3)}
&\gate{V(3,2)_{n(2)\rarrow y_2}}
&\gate{V(3,1)_{n(1)\rarrow y_1}}
&\gate{V(3,0)_{n(0)\rarrow y_0}}
&\gate{\ket{x_{R(3)}}}
}
\end{array}
\;.
\eeq}
\proof
Obvious.
\qed

\claim

\grayeq{
\beq
R=
\begin{array}{c}
\Qcircuit @C=1em @R=1em @!R{
&\uarrowgate\qwx[1]
&\uarrowgate\qwx[2]
&\qw
&\uarrowgate\qwx[3]
&\qw
&\qw
&\qw
\\
&\darrowgate
&\qw
&\uarrowgate\qwx[1]
&\qw
&\uarrowgate\qwx[2]
&\qw
&\qw
\\
&\qw
&\darrowgate
&\darrowgate
&\qw
&\qw
&\uarrowgate\qwx[1]
&\qw
\\
&\qw
&\qw
&\qw
&\darrowgate
&\darrowgate
&\darrowgate
&\qw
}
\end{array}
\;.
\label{eq-r-expan}
\eeq}
\proof
Check that the right hand side
of Eq.(\ref{eq-r-expan})
maps
$0\rarrow 3$,
$1\rarrow 2$,
$2\rarrow 1$, and
$3\rarrow 0$.
\qed
